\documentclass[aspectratio=169]{beamer}

\beamertemplatedotitem
\beamertemplatenavigationsymbolsempty

\usepackage{listings}
\lstdefinelanguage{TRLC}{
  keywords = {abs, and, checks, enum, error, extends, false, fatal, forall, implies, import, in, not, null, optional, or, package, section, true, type, warning, xor},
  comment=[l]{//},
  morecomment=[s]{/*}{*/},
  showstringspaces=false,
  string=[b]{"},
  delim=[s][stringstyle]{'''}{'''}
}

\lstdefinelanguage{SMTLIB}{
  alsoletter=-,
  keywords = {
    and,
    assert,
    check-sat,
    declare-const,
    define-const,
    false,
    get-value,
    or,
    set-logic,
    set-option,
    true,
  },
  moredelim = [s][\color{green!30!black}]{|}{|},
  comment=[l]{;},
  showstringspaces=false,
  string=[b]{"}
}

\lstdefinestyle{smallstyle}
   {basicstyle=\scriptsize\tt,
    keywordstyle=\color{structure},
    commentstyle=\rmfamily\it\color{black!40},
    stringstyle=\color{brown!50!black},
    captionpos=b,
    caption={},label={},
    numbers=none}

\lstset{style=smallstyle}

\usepackage{tikz}
\usetikzlibrary{shadows}

\author{Florian Schanda}
\title{TRLC Static Checker}

\begin{document}

\maketitle

\section{Introduction}
\begin{frame}{Motivation}
  \begin{itemize}
  \item TRLC check expressions are executable
  \item TRLC language defines certain run-time errors
    \begin{itemize}
    \item Null dereference
    \item Division by zero
    \item Array out-of-bounds access
    \item Arithmetic over- and underflow\footnote{Not checked (yet)
        because reference implementation uses arbitrary precision
        arithmetic.}
    \end{itemize}
  \item The Python reference implementation is safe (i.e. throws errors)
  \item A fast C implementation might not be
  \end{itemize}
\end{frame}

\begin{frame}[fragile]{Example}
  \begin{lstlisting}[language=TRLC,gobble=4]
    type Requirement {
      description     optional String
      safety_relevant          Boolean
    }

    checks Requirement {
      len(description) >= 10, "description too short"
    }
  \end{lstlisting}
  \pause
  \begin{scriptsize}
\begin{verbatim}
$ trlc.py --verify foo.rsl
len(description) >= 10, "description too short"
    ^^^^^^^^^^^ foo.rsl:9: issue: expression could be null [vcg-evaluation-of-null]
              | example record_type triggering error:
              |   Requirement bad_potato {
              |     /* description is null */
              |     safety_relevant = false
              |   }
\end{verbatim}
  \end{scriptsize}
\end{frame}

\begin{frame}[fragile]{Is this really so hard?}
  \begin{itemize}
  \item This seems obvious enough...
  \item We could detect that and say ``always prefix with implies''
    \pause
  \item But it gets tricky complex quickly
    \begin{lstlisting}[language=TRLC,gobble=6]
      type Requirement {
        top_level            Boolean
        description optional String
      }

      checks Requirement {
        top_level implies description != null,
          "top level requirements need a description"
      }

      type Top_Level_Requirement extends Requirement {
        freeze top_level = true
      }

      checks Top_Level_Requirement {
        len(description) >= 10, "too short"
      }
    \end{lstlisting}
  \end{itemize}
\end{frame}

\begin{frame}[fragile]{Any complicated semantics can surprising}
  \begin{lstlisting}[language=TRLC,gobble=4]
    Top_Level_Requirement bad_potato {}
  \end{lstlisting}
  Will produce:
  \begin{scriptsize}
\begin{verbatim}
$ trlc.py foo.rsl foo.trlc
Top_Level_Requirement bad_potato {
                      ^^^^^^^^^^ foo.trlc:3: check error: top level requirements
                                             need a description
Top_Level_Requirement bad_potato {
                      ^^^^^^^^^^ foo.trlc:3: error: input to unary expression
                                             len(description) (foo.rsl:18) must not be null
\end{verbatim}
  \end{scriptsize}

  \pause

  Should have made the first check a \structure{fatal} check to
  prevent execution!
\end{frame}

\section{Architecture and background}

\subsection{Linter}
\begin{frame}{Linter}{What is it}
  \begin{itemize}
  \item Formal verification tool
  \item Covers \structure{all possible inputs}, for \structure{all
      possible TRLC implementations}
  \item Models TRLC types and semantics in SMTLIB
  \item Generates counter-examples or proofs of absence of run-time
    errors
  \end{itemize}
\end{frame}

\subsection{Terminology}
\begin{frame}{Terminology}
  \begin{description}
  \item[SAT] (Boolean) satisfiability problem (NP-hard)
  \item[NP-hard] Problem where checking a solution is fast but
    computing a solution is non-polynomial in complexity
    (e.g. $O(2^n)$)
  \item[Undecidable] Problem that cannot be solved by any algorithm
  \item[SMT] SAT modulo theory, an extension of SAT with theories like
    integer or float arithmetic
  \item[SMT Solver] Tool to automatically solve SMT problems
  \item[SMTLIB] Language to describe problems to an SMT solver
  \item[VC] Verification condition (problem you need to solve to
    demonstrate something, e.g. absence of run-time errors)
  \end{description}
\end{frame}

\begin{frame}{Terminology II}
  \begin{description}[Under-approximate]
  \item[Sound] Reasoning that does not miss bugs (i.e. no
    false negatives)
  \item[Complete] Reasoning that does not have false alarms
  \item[Automatic] Reasoning that does not require human intelligence
    as input
  \item[Over-approximate] Analysis that is \structure{sound} and
    \structure{automatic}
  \item[Under-approximate] Analysis that is \structure{complete} and
    \structure{automatic}
  \item[Deductive] Analysis that is \structure{sound} and
    \structure{complete}
  \end{description}
\end{frame}

\subsection{Dependencies}
\begin{frame}{Linter}{Building blocks}
  This would be impossible to just build from scratch, so we use tools:
  \begin{itemize}
  \item PyVCG (a low-level verification condition generator, built for
    TRLC initially but could be useful elsewhere)
  \item CVC5 (a state of the art SMT solver)
  \end{itemize}
\end{frame}

\subsection{Dataflow}
\begin{frame}{Linter}{Dataflow}
  \begin{center}
    \begin{tikzpicture}
      \draw[structure,thick,dashed] (1.5, -3) -- (1.5, 3);
      \draw[structure,thick,dashed] (10.5, -3) -- (10.5, 3);
      \node[align=center] (src) at (0, 0) {Types and\\Checks};
      \node[align=center] (graph) at (3, 0) {Execution\\Graph};
      \node (vc1) at (6, 2) {SMTLIB VC};
      \node (vc2) at (6, 1) {SMTLIB VC};
      \node (vc3) at (6, 0) {SMTLIB VC};
      \node (vc4) at (6, -1) {SMTLIB VC};
      \node (vc5) at (6, -2) {...};
      \node (cvc5) at (9, 0) {CVC5};
      \node (msg) at (12, 0) {Messages};
      \draw[->] (src) -- (graph);
      \draw[->] (graph) -- (vc1);
      \draw[->] (graph) -- (vc2);
      \draw[->] (graph) -- (vc3);
      \draw[->] (graph) -- (vc4);
      \draw[->] (graph) -- (vc5);
      \draw[->] (vc1) -- (cvc5);
      \draw[->] (vc2) -- (cvc5);
      \draw[->] (vc3) -- (cvc5);
      \draw[->] (vc4) -- (cvc5);
      \draw[->] (vc5) -- (cvc5);
      \draw[->] (cvc5) -- (msg);
      \node[align=center,font=\scriptsize,structure] at (0, -3) {TRLC\\(vcg.py)};
      \node[align=center,font=\scriptsize,structure] at (6, -3) {PyVCG};
      \node[align=center,font=\scriptsize,structure] at (12, -3) {TRLC\\(vcg.py, errors.py)};
    \end{tikzpicture}
  \end{center}
\end{frame}

\subsection{SMTLIB}

\begin{frame}[fragile]{SMTLIB}{Some examples}
  \begin{columns}
    \begin{column}{0.5\textwidth}
      \begin{lstlisting}[language=SMTLIB]
(set-logic QF_NIA)
(set-option :produce-models true)

(declare-const a      Int)
(declare-const b      Int)
(declare-const result Int)

(assert (= result
	   (* a b)))

(check-sat)
(get-value (a))
(get-value (b))
(get-value (result))
\end{lstlisting}
\end{column}
    \pause
    \begin{column}{0.5\textwidth}
      \begin{scriptsize}
\begin{verbatim}
sat
((a 0))
((b 0))
((result 0))
\end{verbatim}
      \end{scriptsize}
    \end{column}
  \end{columns}
\end{frame}

\begin{frame}[fragile]{SMTLIB}{Some examples}
  \begin{columns}
    \begin{column}{0.5\textwidth}
      \begin{lstlisting}[language=SMTLIB]
(set-logic QF_NIA)
(set-option :produce-models true)

(declare-const a      Int)
(declare-const b      Int)
(declare-const result Int)

(assert (= result
	   (* a b)))
(assert (= result 42))

(check-sat)
(get-value (a))
(get-value (b))
(get-value (result))
\end{lstlisting}
\end{column}
    \pause
    \begin{column}{0.5\textwidth}
      \begin{scriptsize}
\begin{verbatim}
sat
((a (- 2)))
((b (- 21)))
((result 42))
\end{verbatim}
      \end{scriptsize}
    \end{column}
  \end{columns}
\end{frame}

\begin{frame}[fragile]{SMTLIB}{Some examples}
  \begin{columns}
    \begin{column}{0.5\textwidth}
      \begin{lstlisting}[language=SMTLIB]
(set-logic QF_NIA)
(set-option :produce-models true)

(declare-const a      Int)
(declare-const b      Int)
(declare-const result Int)

(assert (= result
	   (* a b)))
(assert (= result 42))
(assert (= a 4))

(check-sat)
(get-value (a))
(get-value (b))
(get-value (result))
\end{lstlisting}
\end{column}
    \pause
    \begin{column}{0.5\textwidth}
      \begin{scriptsize}
\begin{verbatim}
unsat
\end{verbatim}
      \end{scriptsize}
    \end{column}
  \end{columns}
\end{frame}

\begin{frame}[fragile]{SMTLIB}{How to prove something}
  Lets say we want to prove that $x + 1 > x$:
  \begin{itemize}
  \item First declare variables:
    \begin{lstlisting}[language=SMTLIB,gobble=4]
      (declare-const x Int)
    \end{lstlisting}
    \pause
  \item Then define a goal:
    \begin{lstlisting}[language=SMTLIB,gobble=4]
      (define-const goal Bool
        (> (+ x 1)
           x))
    \end{lstlisting}
    \pause
  \item Then assert that the goal is not true:
    \begin{lstlisting}[language=SMTLIB,gobble=4]
      (assert (not goal))
    \end{lstlisting}
  \item Then ask for a model:
    \begin{lstlisting}[language=SMTLIB,gobble=4]
      (check-sat)
    \end{lstlisting}
    \pause
  \item If we get a model: we know it's not (always) true and we have
    a specific \structure{counter-example}
    \pause
  \item If we don't: we know there are no counter-examples, i.e. the
    original goal is always true
  \end{itemize}
\end{frame}

\section{TRLC execution semantics}

\begin{frame}{TRLC}{Execution semantics}
  \begin{itemize}
  \item Mostly just expressions (e.g. $len(x) + len(y) > 10$)
  \item Control flow is rare, but we have some:
    \begin{itemize}
    \item and, or, and implies (short-circuit semantics)
    \item range tests
    \item if expressions
    \item ordering of (fatal) checks inside a block
    \item checks from parent types before checks from extension
    \end{itemize}
  \item Interesting cases:
    \begin{itemize}
    \item Execution order from checks from different blocks is unspecified
    \item Execution order inside quantifiers is unspecified
    \item Execution continues after (non-fatal) errors and warnings
    \end{itemize}
  \end{itemize}
\end{frame}

\begin{frame}[fragile]{TRLC}{Execution semantics example}
  \begin{lstlisting}[language=TRLC,gobble=4]
    x != null implies x in 1 .. 10, "potato"
  \end{lstlisting}
  \begin{center}
    \begin{tikzpicture}
      \node(start) at (0, 0) {start};
      \node(e1) at (3, 0) {$x \neq null$};
      \node(e2) at (6, -1.5) {$x \geq 1$};
      \node(e3) at (9, -3) {$x \leq 10$};
      \node(end) at (12, 0) {end};

      \draw[->] (start) -- (e1);
      \draw[->] (e1) -- node[above] {$\top$} (e2);
      \draw[->] (e1) -- node[above] {$\bot$} (end);
      \draw[->] (e2) -- node[above] {$\top$} (e3);
      \draw[->] (e2) -- node[above] {$\bot$} (end);
      \draw[->] (e3) -- (end);

      \visible<2-> {
        \node[structure,font=\scriptsize,align=left] (check1) at (6, -2.5) {
          RTE check:\\
          $\lnot (x = null)$
        };
        \draw[structure] (e2) -- (check1);
        \node[structure,font=\scriptsize,align=left] (check2) at (9, -4) {
          RTE check:\\
          $\lnot (x = null)$
        };
        \draw[structure] (e3) -- (check2);
      }

      \visible<3-> {
        \node[structure,font=\scriptsize,align=left] (check3) at (12, 1) {
          feasibility check:\\
          $\exists x \mid \lnot (x \neq null \implies 1 \leq x \leq 10)$
        };
        \draw[structure] (end) -- (check3);
      }
    \end{tikzpicture}
  \end{center}
\end{frame}

\section{TRLC VCG}
\begin{frame}{TRLC}{Required tasks}
  \begin{itemize}
  \item Build execution graph and model TRLC (TRLC with PyVCG)
  \item Annotate this graph with checks (TRLC with PyVCG)
  \item Generate VCs (PyVCG)
  \item Solve them (PyVCG with CVC5)
  \item Generate feedback to the user (TRLC with PyVCG)
  \end{itemize}
\end{frame}

\begin{frame}{TRLC}{PyVCG}
  \begin{itemize}
  \item PyVCG does all the graph and SMT stuff
  \item I decided to factor it out since it could be reusable for
    other projects
  \item I decided not to use Why3\footnote{https://why3.lri.fr} or
    Boogie\footnote{https://www.microsoft.com/en-us/research/project/boogie-an-intermediate-verification-language}
    because I don't like quantifiers
  \item I decided not to use
    GOTO\footnote{https://www.cprover.org/cbmc/} because its somewhat
    annoying to work with and I \emph{do} need quantifiers and
    infinite integers
  \item (Also, I was extremely bored)
  \end{itemize}
\end{frame}

\section{Modelling TRLC}
\begin{frame}{Modelling TRLC}{Why}
  TRLC $\neq$ SMTLIB:
  \begin{itemize}
  \item TRLC is sequential and executable
  \item SMTLIB is declarative
  \item Types are different
  \item There are no ``null'' values in SMTLIB
  \end{itemize}
\end{frame}

\subsection{Modelling null}
\begin{frame}[fragile]{Modelling TRLC}{Null}
  \begin{itemize}
  \item Optional components can be ``null''
  \item Expressions generally can't be (\verb|x + 1| itself cannot be
    null, but \verb|x| could be)
  \item Quantification variables can't be (in
    \verb|(forall x in y => x)| the bound variable \verb|x| cannot be
    null, but \verb|y| could be)
  \item Array members can't be (although the whole array itself could
    be)
  \end{itemize}
\end{frame}

\begin{frame}[fragile]{Modelling TRLC}{Null}
  I considered two options:
  \begin{itemize}
  \item Make a datatype for everything: $x = (isNull, actualValueOfX)$
    \begin{description}
    \item[+] Very generic, easy to get right
    \item[-] Requires re-implementation of all operations
    \item[-] Has the smell of three-valued logic
    \item[-] Might make quantifiers worse
    \end{description}
  \item Make a separate value for everything: $x = actualValueofX, valid\_x$
    \begin{description}
    \item[-] Requires more manual work when generating VCs
    \item[+] Probably way faster if you can optimise it away
    \end{description}
  \end{itemize}
  I chose the second option.
\end{frame}

\begin{frame}{Modelling TRLC}{Null}
  How this looks like with $x = y$:\pause
  \begin{itemize}
  \item There are four variables here:
    \begin{itemize}
    \item $x$ and $y$
    \item $valid\_x$ and $valid\_y$
    \end{itemize}
    \pause
  \item Semantics of equality are:
    \begin{description}
    \item[Equality\_On\_Null] Null is only equal to itself.
    \item[Simple\_Relational\_Semantics] The meaning of the relationship
      operators are the usual.
    \end{description}
    \pause
  \item $valid\_x = valid\_y \land (valid\_x \implies x = y)$
  \end{itemize}
\end{frame}

\begin{frame}{Modelling TRLC}{Null}
  \begin{itemize}
  \item So $x = y$ is
    $valid\_x = valid\_y \land (valid\_x \implies x = y)$
    \pause
  \item If we know that $y$ is not optional, we could simplify:
    \begin{itemize}
    \item Flip around:
      $valid\_y = valid\_x \land (valid\_y \implies y = x)$
      \pause
    \item Substitute $\top$ for $valid\_y$:
      $\top = valid\_x \land (\top \implies y = x)$
      \pause
    \item Simplify:
      $valid\_x \land y = x$
    \end{itemize}
    \pause
  \item If we know that $x$ is also not optional, we could simplify
    further: $x = y$
  \end{itemize}
\end{frame}

\begin{frame}{Modelling TRLC}{Null}
  Final strategy:
  \begin{itemize}
  \item Introduce a new Boolean for each value indicating the validity
  \item Perform arithmetic on just the values, assuming validity is OK
  \item Perform validity checks on just the validity values
  \item Only place we mix is for the {\tt ==} and {\tt !=} operators
  \item Most of it can be simplified
  \end{itemize}
\end{frame}

\begin{frame}[fragile]{Modelling TRLC}{Null}
  A real example:
  \begin{lstlisting}[language=TRLC,gobble=4]
    type T {
      x optional Boolean
      y optional Boolean
    }
    checks T {
      x == y, "potato"
    }
  \end{lstlisting}
\end{frame}

\begin{frame}[fragile]{Modelling TRLC}{Null}
  A real example:
  \begin{lstlisting}[language=SMTLIB,gobble=4]
    ;; value for x declared on foo.rsl:3:3
    (declare-const |Foo.T.x.value| Bool)
    (declare-const |Foo.T.x.valid| Bool)
    ;; value for y declared on foo.rsl:4:3
    (declare-const |Foo.T.y.value| Bool)
    (declare-const |Foo.T.y.valid| Bool)
    ;; result of x == y at foo.rsl:7:5
    (define-const |tmp.1| Bool
      (and (= |Foo.T.x.valid| |Foo.T.y.valid|)
           (=> |Foo.T.x.valid| (= |Foo.T.x.value| |Foo.T.y.value|))))
  \end{lstlisting}
\end{frame}

\subsection{Modelling the execution}
\begin{frame}[fragile]{Modelling TRLC}{Execution graph and null}
  One more example:
  \begin{lstlisting}[language=TRLC,gobble=4]
    type T {
      x optional Integer
    }
    checks T {
      x + 1 > x, "potato"
    }
  \end{lstlisting}
  \pause
  Requires more than one VC:
  \begin{itemize}
  \item VC1: Validity check for x on LHS
  \item We can then compute the value of $x + 1$ assuming it's valid
    \pause
  \item VC2: Validity check for x on RHS
  \item We can then compute the value of $x + 1 > x$ assuming it's valid
    \pause
  \item We can now use the computed values to do something like check
    if it's always true
  \end{itemize}
\end{frame}

\begin{frame}[fragile]{Modelling TRLC}{Execution graph and null}
  \begin{columns}
    \begin{column}{0.5\textwidth}
      \begin{center}
        \begin{tikzpicture}[every node/.style={font=\scriptsize,align=center}]

          \node(start) at (0, 0.25) {start};
          \node(a1) at (0, -0.75) {
            {\bf Assumption}\\
            (declare-const $\mid$Foo.T.x.value$\mid$ Int)\\
            (declare-const $\mid$Foo.T.x.valid$\mid$ Bool)
          };
          \node<1,3->(c1) at (0, -2) {
            {\bf Check}\\
            goal: $\mid$Foo.T.x.valid$\mid$
          };
          \node<2>[structure](c1) at (0, -2) {
            {\bf Check}\\
            goal: $\mid$Foo.T.x.valid$\mid$
          };
          \node(a2) at (0, -3) {
            {\bf Assumption}\\
            (define-const $\mid$tmp.1$\mid$ Int (+ $\mid$Foo.T.x.value$\mid$ 1))
          };
          \node<-2,4->(c2) at (0, -4) {
            {\bf Check}\\
            goal: $\mid$Foo.T.x.valid$\mid$
          };
          \node<3>[structure](c2) at (0, -4) {
            {\bf Check}\\
            goal: $\mid$Foo.T.x.valid$\mid$
          };
          \node(a3) at (0, -5) {
            {\bf Assumption}\\
            (define-const $\mid$tmp.2$\mid$ Bool ($>$ $\mid$tmp.1$\mid$ $\mid$Foo.T.x.value$\mid$))
          };
          \node<-4>(c3) at (-1.5, -6) {
            {\bf Check}\\
            goal: $\mid$tmp.2$\mid$
          };
          \node<4>[structure](c3) at (-1.5, -6) {
            {\bf Check}\\
            goal: $\mid$tmp.2$\mid$
          };
          \node(a4) at (1.5, -6) {
            {\bf Assumption}\\
            (assert $\mid$tmp.2$\mid$)
          };
          \draw[->] (start) -- (a1);
          \draw[->] (a1) -- (c1);
          \draw[->] (c1) -- (a2);
          \draw[->] (a2) -- (c2);
          \draw[->] (c2) -- (a3);
          \draw[->] (a3) -- (c3);
          \draw[->] (a3) -- (a4);
        \end{tikzpicture}
      \end{center}
    \end{column}
    \begin{column}{0.5\textwidth}
      \begin{onlyenv}<2>
        \begin{lstlisting}[language=SMTLIB,gobble=10]
          ;; value for x declared on foo.rsl:3:3
          (declare-const |Foo.T.x.value| Int)
          (declare-const |Foo.T.x.valid| Bool)
          ;; validity check for x
          (assert (not |Foo.T.x.valid|))
          (check-sat)
        \end{lstlisting}
      \end{onlyenv}
      \begin{onlyenv}<3>
        \begin{lstlisting}[language=SMTLIB,gobble=10]
          ;; value for x declared on foo.rsl:3:3
          (declare-const |Foo.T.x.value| Int)
          (declare-const |Foo.T.x.valid| Bool)
          (assert |Foo.T.x.valid|)
          ;; result of x + 1 at foo.rsl:6:5
          (define-const |tmp.1| Int
            (+ |Foo.T.x.value| 1))
          ;; validity check for x
          (assert (not |Foo.T.x.valid|))
          (check-sat)
        \end{lstlisting}
      \end{onlyenv}
      \begin{onlyenv}<4>
        \begin{lstlisting}[language=SMTLIB,gobble=8]
          ;; value for x declared on foo.rsl:3:3
          (declare-const |Foo.T.x.value| Int)
          (declare-const |Foo.T.x.valid| Bool)
          (assert |Foo.T.x.valid|)
          ;; result of x + 1 at foo.rsl:6:5
          (define-const |tmp.1| Int
            (+ |Foo.T.x.value| 1))
          (assert |Foo.T.x.valid|)
          ;; result of x + 1 > x at foo.rsl:6:9
          (define-const |tmp.2| Bool
            (> |tmp.1| |Foo.T.x.value|))
          ;; feasability check for x + 1 > x
          (assert (not |tmp.2|))
          (check-sat)
        \end{lstlisting}
      \end{onlyenv}
    \end{column}
  \end{columns}
\end{frame}

\subsection{Debugging options}
\begin{frame}{Modelling TRLC}{Debugging options}
  You can see this for real using the {\tt --debug-vcg} option:
  \begin{itemize}
  \item Generates a {\tt .pdf} for the graph
  \item Generates a {\tt .smt2} file for each VC
  \end{itemize}
\end{frame}

\subsection{Modelling types}
\begin{frame}{Modelling TRLC}{Types}
  We have these types in TRLC:
  \begin{itemize}
  \item Boolean \only<2->{(SMTLIB Bool)}
  \item Integer \only<2->{(SMTLIB Int)}
  \item Decimal \only<3->{(SMTLIB Real, over-approximation)}
  \item String (and Markup\_String) \only<4->{(SMTLIB String,
      over-approximation for Markup\_String)}
  \item Enumerations \only<5->{(SMTLIB Datatypes)}
  \item User-defined records \only<6->{(SMTLIB Datatypes)}
  \item User-defined tuples \only<6->{(SMTLIB Datatypes)}
  \item Arrays consisting out of any of the above \only<7->{(SMTLIB Sequence)}
  \end{itemize}
\end{frame}

\subsection{Approximating decimals}

\begin{frame}{Modelling TRLC}{Decimals}
  Decimals are annoying... There are no decimals in SMTLIB.
  \begin{center}
    \begin{equation*}
      \mathbb{D} \in \mathbb{Q} \in \mathbb{R}
    \end{equation*}
  \end{center}
  \pause
  \begin{itemize}
  \item $0.3$ is a decimal
    \pause
  \item $\frac{1}{3}$ is a rational (but not decimal)
    \pause
  \item $\sqrt{2}$ is a real (but not rational, and also not a decimal)
    \pause
  \item There are things true in $\mathbb{D}$ that are not true
    $\mathbb{R}$: $\forall x \in \mathbb{D} \mid x + \frac{1}{3} \neq 0$
    \pause
  \item There are things true in $\mathbb{R}$ that are not true
    $\mathbb{D}$: $\exists x \in \mathbb{R} \mid x * x = 2$
  \end{itemize}
\end{frame}

\begin{frame}{Modelling TRLC}{Decimals}
  Options:
  \begin{itemize}
  \item Beg CVC5 developers for a Decimal extension (I tried)
    \pause
  \item Model as a pair of integers $value = \frac{a}{b}$ and say:
    \begin{itemize}
    \item $\exists n \in \mathbb{N} | n > 0 \land b = 10^n$
    \end{itemize}
    \pause
  \item This is awful because:
    \begin{itemize}
    \item $\mathbb{N} / \mathbb{R}$ conversions
    \item power is basically unsupported
    \item existential quantification
    \item nonlinear division everywhere
    \end{itemize}
  \end{itemize}
\end{frame}

\begin{frame}{Modelling TRLC}{Decimals}
  Options:
  \begin{itemize}
  \item Rational with restricted precision $value = \frac{a}{b}$
    \begin{itemize}
    \item E.g. $b \in \{1, 10, 100, 1000, 10000, 100000\}$ for 5
      decimal digits
    \item This is an under-approximation (i.e. not sound)
    \item disjunctions everywhere
    \end{itemize}
    \pause
  \item \structure{Treat as real} (over-approximation)
    \begin{itemize}
    \item Best performance
    \item You sometimes get impossible counter-examples
    \end{itemize}
    (This is what I chose.)
  \end{itemize}
\end{frame}

\begin{frame}[fragile]{Modelling TRLC}{Decimals}
  Example of an incorrect counter-example:
  \begin{lstlisting}[language=TRLC,gobble=4]
    type T {
      a Decimal
    }

    checks T {
      1.0 / (a + (1.0 / 3.0)) > 0.0, "potato"
    }
  \end{lstlisting}
  \begin{scriptsize}
\begin{verbatim}
1.0 / (a + (1.0 / 3.0)) > 0.0, "potato"
    ^ test2.rsl:8: issue: divisor could be 0.0 [vcg-div-by-zero]
    | example record_type triggering error:
    |   T bad_potato {
    |     a = -1 / 3
    |   }
\end{verbatim}
  \end{scriptsize}
\end{frame}

\end{document}
